%
% Softheme internal lectures
%
% Formal models and methods for program semantics, correctness and verification
%
% (c) Vadim Vinnik, 2014
%
% vadim.vinnik@gmail.com
%

\documentclass[landscape]{slides}
\usepackage[landscape, margin=10mm]{geometry}

\usepackage[russian]{babel}
\usepackage[T2A]{fontenc}
\usepackage[utf8]{inputenc}

\setlength{\unitlength}{6mm}

\usepackage{amssymb}
\usepackage{amsmath}
\newcommand{\lcurlybrace}\{
\newcommand{\rcurlybrace}\}
\newcommand{\lbracket}{[}
\newcommand{\rbracket}{]}
\newcommand{\tripleImpl}[5]{\left#4 #1 \right#5 \,  {#2} \,  \left#4 #3 \right#5}
\newcommand{\tripleP}[3]{\tripleImpl{#1}{#2}{#3}{\lcurlybrace}{\rcurlybrace}}
\newcommand{\tripleT}[3]{\tripleImpl{#1}{#2}{#3}{\lbracket}{\rbracket}}
\newcommand{\true}{\mathbf{True}}
\newcommand{\false}{\mathbf{False}}
\renewcommand{\implies}{\Rightarrow}
\newcommand{\implied}{\Leftarrow}
\newcommand{\code}[1]{\mathtt{#1}}
\newcommand{\nop}{\code{nop}}
\newcommand{\chaos}{\code{chaos}}
\newcommand{\hang}{\code{hang}}
\newcommand{\ifelse}[3]{\code{if}\ {#1}\ \code{then}\ {#2}\ \code{else}\ {#3}\ \code{fi}}
\newcommand{\ifthen}[2]{\code{if}\ {#1}\ \code{then}\ {#2}\ \code{fi}}
\newcommand{\while}[2]{\code{while}\ {#1}\ \code{do}\ {#2}\ \code{od}}
\newcommand{\assign}{\ \code{:=}\ }
\newcommand{\infinum}{\mathrm{inf}}
\newcommand{\supremum}{\mathrm{sup}}
\newcommand{\wpsign}{\mathbf{wp}}
\newcommand{\wpof}[2]{\wpsign\left({#1},\ {#2}\right)}
\newcommand{\modin}{\mathbin{\mathrm{mod}}}


\begin{document}

\author{В.Ю.\,Винник}

\title{%
Формальные модели\\%
семантики и корректности,\\%
методы спецификации\\%
и верификации программ.\\
Часть~2}

\date{Softheme, 17.06.2014 н.э}

\maketitle

\begin{slide}
        \begin{center}
                \begin{LARGE}
                        Теоретизирование~--\\это отыскивание сомнительных причин\\для обоснования того,\\во что и так веришь\\интуитивно.
                \end{LARGE}
        \end{center}
\end{slide}

\begin{slide}
        \begin{center}
                \begin{LARGE}
                        Repetitio est mater studiorum
                \end{LARGE}
        \end{center}
\end{slide}

\begin{slide}
        Программа как чёрный ящик принимает исходные данные и выдаёт результаты

        \begin{center}
                \begin{picture}(8,10)
                        \put(0,8){\makebox(8,2)[c]{\shortstack[c]{Исходные данные, начальное состояние памяти}}}
                        \put(4,8){\vector(0,-1){2}}
                        \put(0,4){\framebox(8,2)[c]{Программа}}
                        \put(4,4){\vector(0,-1){2}}
                        \put(0,0){\makebox(8,2)[c]{\shortstack[c]{Результаты, заключительное состояние памяти}}}
                \end{picture}
        \end{center}

        Описать семантику программы~-- значит описать свойства отображения исходных
        данных на результаты.

        Задать спецификацию программы~-- значит описать желаемое соотношение между
        исходными данными и результатами.

        Описать семантику управляющей конструкции~-- значит показать, как строится
        семантика составной программы из семантики программ, охваченных ею.
\end{slide}

\begin{slide}
        Метод Ч.\,Э.\,Р.\,Хоара (метод хоаровских троек)
        \begin{itemize}
                \item Предусловие~-- налагаемое на начальное состояние памяти.
                \item Постусловие~-- налагаемое на заключительное состояние.
                \item Спецификация~-- предусловие и постусловие.
                \item Пред- и постусловие выражены предикатами, в которые входят
                        имена переменных программы.
        \end{itemize}
\end{slide}

\begin{slide}
        Корректность программы относительно спецификации:
        \begin{itemize}
                \item Полная: если начальное состояние памяти удовлетворяет предусловию,
                        то программа завершается, причём её заключительное состояние
                        удовлетворяет постусловию.
                        \[
                                \tripleT{P}{S}{Q}.
                        \]
                \item Частичная: если начальное состояние памяти удовлетворяет предусловию,
                        и если программа завершается, то её заключительное состояние
                        удовлетворяет постусловию.
                        \[
                                \tripleP{P}{S}{Q}.
                        \]
                \item Если исходные данные нарушают предусловие, программа имеет право
                делать что угодно, никакое её поведение не подтвердит и не опровергнет
                её корректность.
        \end{itemize}
\end{slide}

\begin{slide}
        \begin{itemize}
                \item Частичная корректность~-- программа имеет право не дать
                        никакого результата, лишь бы не дала неправильный.
                \item Частичная корректность~-- ни один запуск программы не даст свидетельства
                        её некорректности (презумпция невиновности).
                \item Программа, зависающая на любых исходных данных, корректна (в частичном смысле)
                        относительно любой спецификации.
                \item Всегда зависающая программа (в модели частичной корректности)~-- это универсальный
                        решатель любых задач!
                \item Если такая программа завершится, она выполнит любое желание. Но она не завершится
                        никогда, поэтому её нельзя будет поймать на невыполнении желания.
        \end{itemize}
\end{slide}

\begin{slide}
Примеры спецификаций
\begin{itemize}
\item Нахождение максимума $c$ из значений двух переменных $a$, $b$:
        \[
                \tripleT
                        {a = \alpha \land b = \beta}
                        {S}
                        {
                                c \geqslant \alpha \land
                                c \geqslant \beta \land
                                (c = \alpha \lor c = \beta)
                        } .
        \]
\item Обмен значениями переменных $a$ и $b$
        \[
                \tripleT
                        {a = \alpha \land b = \beta}
                        {S} 
                        {a = \beta \land b = \alpha} .
        \]
\item Нахождение наибольшего общего делителя $c$ значений переменных $a$, $b$:
        \[
                \tripleT
                        {a = \alpha \land b = \beta}
                        {S} 
                        {
                                P(c) \land
                                \forall \gamma : P(\gamma) \implies \gamma \leqslant c
                        } ,
        \]
        где через $P(\xi)$ обозначен предикат "<$\xi$ делит $\alpha$ и $\beta$">:
        \[
                P(\xi) = (\exists \delta, \varepsilon : \alpha = \delta\cdot \xi \land \beta = \varepsilon\cdot \xi).
        \]
\end{itemize}
\end{slide}

\begin{slide}
Общие свойства спецификаций и корректности
\begin{itemize}
\item Требование завершимости программы~$S$ на исходных данных из~$P$:
        \[
                \tripleT{P}{S}{\true} .
        \]
\item Требование незавершимости:
        \[
                \tripleP{P}{S}{\false} .
        \]
\item Матмодель частичной корректности удобнее, поэтому анализ полной коррректности
        разбивают на две задачи: доказательство частичной корректности и отдельно
        доказательство завершимости:
        \[
                \left(\tripleP{P}{S}{Q}\right)
                \land 
                \left(\tripleT{P}{S}{\true}\right)
                \implies
                \left(\tripleT{P}{S}{Q}\right) .
        \]
\end{itemize}
\end{slide}

\begin{slide}
\begin{itemize}
\item Тавтологии~-- тождественно истинные спецификации:
        \[
                \tripleP{P}{S}{\true} ,
                \quad
                \tripleP{\false}{S}{Q} .
        \]
\item Тождественно ложно для любого $P\neq \false$:
        \[
                \tripleT{P}{S}{\false}
        \]
\item Законы консеквенции
        \[
                \frac
                {
                        P \implies R, \  \tripleP{R}{S}{Q}
                }
                {
                        \tripleP{P}{S}{Q}
                } ,
                \qquad
                \frac
                {
                        \tripleP{P}{S}{R}, \  R \implies Q
                }
                {
                        \tripleP{P}{S}{Q}
                } ,
        \]
        усиление пред- и ослабление постусловия.
\item Законы дистрибутивности
        \[
                \frac
                {
                        \tripleP{P_1}{S}{Q}, \  \tripleP{P_2}{S}{Q}
                }
                {
                        \tripleP{P_1 \lor P_2}{S}{Q}
                },
                \qquad
                \frac
                {
                        \tripleP{P}{S}{Q_1}, \  \tripleP{P}{S}{Q_2}
                }
                {
                        \tripleP{P}{S}{Q_1 \land Q_2}
                },
        \]
        работают в обе стороны~-- в силу законов консеквенции
\end{itemize}
\end{slide}
  
\begin{slide}
Две самые важные программы
\begin{itemize}
\item Пустой оператор: для любого $P$
        \[
                \tripleP{P}{\nop}{P} .
        \]
\item Зависание: для любого $P$
        \[
                \tripleP{P}{\hang}{\false} .
        \]
\end{itemize}
        Эквивалентность программ по поведению: $S_1 = S_2$ по определению, если для любых~$P$, $Q$
        \[
                \tripleP{P}{S_1}{Q}
                \ \Leftrightarrow\  
                \tripleP{P}{S_2}{Q}
        \]
\end{slide}

\begin{slide}
        Последовательная композиция
        \begin{center}
                \begin{picture}(18,2)
                        \put(0,0){\makebox(2,2)[c]{\shortstack[c]{$x_0$}}}
                        \put(2,1){\vector(1,0){2}}
                        \put(4,0){\framebox(2,2)[c]{$S_1$}}
                        \put(6,1){\vector(1,0){2}}
                        \put(8,0){\makebox(2,2)[c]{\shortstack[c]{$x_1$}}}
                        \put(10,1){\vector(1,0){2}}
                        \put(12,0){\framebox(2,2)[c]{$S_2$}}
                        \put(14,1){\vector(1,0){2}}
                        \put(16,0){\makebox(2,2)[c]{\shortstack[c]{$x_2$}}}
                \end{picture}
        \end{center}
 
                Результат первой программы~-- вход второй. Условие, наложенное на промежуточное состояние,
                есть постусловие для~$S_1$ и предусловие для~$S_2$:
                \[
                        \frac
                        {
                                \tripleP{P}{S_1}{Q}
                                \quad
                                \tripleP{Q}{S_2}{R}
                        }
                        {
                                \left\{P\right\}
                                S_1 ; S_2
                                \left\{R\right\}
                        }
                \]

        Теорема об ассоциативности последовательной композиции
        \[
                (S_1;S_2);S_3
                =
                S_1;(S_2;S_3)
        \]
        Аналог ассоциативности сложения:
        \[
                (a + b) + c = a + (b + c).
        \]
\end{slide}

\begin{slide}
        Доказательство ($\Rightarrow$). Пусть $\tripleP{P}{(S_1;S_2);S_3}{Q}$. Тогда существует такое~$T$, что
        \[
                \tripleP{P}{S_1;S_2}{T},\ (1)
                \qquad
                \tripleP{T}{S_3}{Q}.\ (2)
        \]
        Из (1) следует существование такого~$U$, что
        \[
                \tripleP{P}{S_1}{U},\ (3)
                \qquad
                \tripleP{U}{S_2}{T},\ (4)
        \]
        Из (4) и (2) следует
        \[
                \tripleP{U}{S_2;S_3}{Q}. \ (5)
        \]
        Теперь из (3) и (5) следует
        \[
                \tripleP{P}{S_1; (S_2; S_3)}{Q} ,
        \]
        Quod erat demonstrandum.

        Доказательство ($\Leftarrow$) аналогично.
\end{slide}

\begin{slide}
        Особые свойства операторов относительно последовательной композиции
        \begin{itemize}
                \item Пустой оператор, поскольку $\triple{R}{\nop}{R}$:
                        \[
                                \nop ; S = S ; \nop = S .
                        \]

                \item Арифметический аналог:
                \[
                        1\cdot x = x\cdot 1 = x
                \]

                \item Зависание, поскольку $\triple{R}{\hang}{\false}$:
                        \[
                                S ; \hang = \hang .
                        \]

                \item Арифметический аналог:
                \[
                        0\cdot x = x\cdot 0 = 0
                \]

                \item Не выполняется свойство левого нуля, т.к. $\triple{\false}{S}{Q}$:
                        \[
                                \hang ; S \neq \hang .
                        \]
        \end{itemize}
\end{slide}

\begin{slide}
        Ветвление
        \begin{center}
                \begin{picture}(24,9)(-12,-9)
                        \put(-2,0){\makebox(4,1)[c]{\shortstack[c]{$\{P\}$}}}
                        \put(0,0){\vector(0,-1){1}}
                        \put(-4,-2){\line(4,1){4}}
                        \put(-4,-2){\line(4,-1){4}}
                        \put(4,-2){\line(-4,-1){4}}
                        \put(4,-2){\line(-4,1){4}}
                        \put(-4,-4){\makebox(8,4)[c]{\shortstack[c]{$b$}}}
                        \put(-5,-0.5){\makebox(0,0)[rt]{{$\{P \land b\}$}}}
                        \put(5,-0.5){\makebox(0,0)[lt]{{$\{P \land \neg b\}$}}}
                        \put(-4,-2){\line(-1,0){4}}
                        \put(-8,-2){\vector(0,-1){2}}
                        \put(-12,-6){\framebox(8,2)[c]{\shortstack[c]{$S_1$}}}
                        \put(-8,-6){\line(0,-1){2}}
                        \put(-8,-8){\line(1,0){8}}
                        \put(4,-2){\line(1,0){4}}
                        \put(-2,-8.9){\makebox(4,4)[c]{\shortstack[c]{$\{Q\}$}}}
                        \put(8,-2){\vector(0,-1){2}}
                        \put(4,-6){\framebox(8,2)[c]{\shortstack[c]{$S_2$}}}
                        \put(8,-6){\line(0,-1){2}}
                        \put(8,-8){\line(-1,0){8}}
                        \put(0,-8){\vector(0,-1){1}}
                \end{picture}
        \end{center}
        Выполняется программа от начала к концу, а анализируется~-- с конца к началу!
        \[
                \frac
                {
                        \tripleP{P \land b}{S_1}{Q}
                        \quad
                        \tripleP{P \land \neg b}{S_2}{Q}
                }
                {
                        \tripleP{P}{\ifelse{b}{S_1}{S_2}} {Q}
                }
        \]
\end{slide}

\begin{slide}
        Теоремы об эквивалентных преобразованиях

Идемпотентность:
        \[
                S = \ifelse{b}{S}{S} .
        \]

Сокращение:
        \[
                S = \ifelse{\true}{S}{T} .
        \]

Обращение условия:
        \[
                \ifelse{b}{S_1}{S_2} = \ifelse{\neg b}{S_2}{S_1} .
        \]

Дистрибутивность последовательной относительно условной:
        \[
                S_1; \ifelse{b}{S_2}{S_3} = \ifelse{b}{S_1; S_2}{S_1 ;S_3} .
        \]

Перестановка условий:
        \begin{multline*}
                \ifelse{b}{(\ifelse{c}{S_1}{S_2})}{S_3} = \\
                \ifelse{c}{(\ifelse{b}{S_1}{S_3})}{(\ifelse{b}{S_2}{S_3})} .
        \end{multline*}
\end{slide}

\begin{slide}
        Циклическая композиция

        Инвариант цикла~-- предикат~$I$, истинность которого сохраняется телом цикла.
        
        Если истинность инварианта обеспечена перед входом в цикл, то она
        гарантированно сохраняется после каждой итераци, а значит и после завершения цикла.

        Условие повторения~$b$ необходимо ложно после завершения цикла, ведь в противном случае цикл
        бы не мог завершиться.
        
        \[
                \frac
                        {\tripleP{b\land I}{S}{I}}
                        {\tripleP{I}{\while{b}{S}}{\neg b\land I}}
        \]

        Пусть $t\geqslant 0$~-- натуральнозначная функция от состояния
        \[
                \frac
                        {\tripleT{I \land b \land t=\tau}{S}{I \land t < \tau}}
                        {\tripleT{I}{\while{b}{S}}{\neg b\land I}}
        \]
\end{slide}

\begin{slide}
        Замечательные свойства циклов:

        \begin{itemize}
        \item Пустой цикл
        \[
                \while{\false}{S} = \nop
        \]
        \item Вечный цикл
        \[
                \while{\true}{S} = \hang
        \]
        \item Поглощение условия внутри
        \[
                \while{b}{(\ifthen{b}{S})} = \while{b}{S}
        \]
        \item Поглощение условия снаружи
        \[
                \ifthen{b}{(\while{b}{S})} = \while{b}{S}
        \]
        \item Неподвижная точка
        \[
                \ifthen{b}{S} ; \while{b}{S} = \while{b}{S}
        \]
        \end{itemize}
\end{slide}

\begin{slide}
\end{slide}

\begin{slide}
\end{slide}

\begin{slide}
\end{slide}

\begin{slide}
        \begin{Large}
Теория — это когда все известно, но ничего не работает. Практика — это когда все работает, но никто не знает почему. Мы же объединяем теорию и практику: ничего не работает, и никто не знает почему!
        \end{Large}

\begin{flushright}
\emph{Альберт Эйнштейн}
\end{flushright}
\end{slide}

\begin{slide}
        Литература
        \begin{itemize}
\item Hoare, C.\,A.\,R. An axiomatic basis for computer programming.\\Communications of the ACM 12 (10): 576–580.
\item Floyd, R.\,W. Assigning meanings to programs. Proceedings of the American Mathematical Society Symposia on Applied Mathematics. Vol. 19, pp. 19–31. 1967.
\item Hoare, C.\,A.\,R. and He Jifeng, Unifying Theories of Programming, Prentice Hall International Series in Computer Science, 1998.\\http://unifyingtheories.org
        \end{itemize}
\end{slide}

\begin{slide}
        \begin{itemize}
\item Дейкстра Э. Дисциплина программирования. — М.: Мир, 1978. — 280 с.
\item Грис Д. Наука программирования. — М.: Мир, 1984. — 289 с.
\item Алагич С., Арбиб М. Проектирование корректных структурированных программ. — М.: Радио и связь, 1984. — 264 с.
\item Андерсон Р. Доказательство правильности программ. — М.: Мир, 1982 — 287 с.
\item Калинин А.Г., Мацкевич И.В. Универсальные языки программирования. Семантический подход. — М.: Радио и связь, 1991. — 400 с.
        \end{itemize}
\end{slide}

\begin{slide}
        \begin{itemize}
\item Абрамов С.А  Элементы анализа программ. Частичные функции на множестве состояний. — М.: Наука, 1986. — 128 с.
\item Замулин А.В. Формальная модель Java-программы, основанная на машинах абстрактных состояний // Программирование. — 2003. — № 3. — С. 15–28.
\item Замулин А.В. Формальная семантика выражений и операторов языка Java // Программирование. — 2003. — № 5. — С. 31–45.
\item Замулин А.В. Алгебраическая семантика императивного языка программирования // Программирование. — 2003. — № 6. — С. 51–64. 
\item Egon Börger and Robert Stärk, Abstract State Machines: A Method for High-Level System Design and Analysis. Springer-Verlag, 2003.
        \end{itemize}
\end{slide}

\end{document}
